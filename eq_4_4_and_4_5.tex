\documentclass[10pt,notitlepage,letterpaper,landscape,twocolumn]{article}

\usepackage[T1]{fontenc}
%\usepackage{textcomp}
\usepackage{bookman}
%\usepackage{color}
%\usepackage{graphicx}
\newcommand{\besY}{\mathrm{Y}}
\newcommand{\rot}{\vec{\mathrm{rot}}}
\newcommand{\ez}{\vec{\mathrm{e}}_z}
\newcommand{\eg}{\vec{\mathrm{e}}_\Gamma}
\newcommand{\en}{\vec{\mathrm{e}}_n}
\newcommand{\nabt}{\vec{\nabla}_t}
\newcommand{\eqnr}[1]{(\ref{eqn:#1})}
\newcommand{\dd}{\mathrm{d}}

\setlength{\textheight}{1.05\textheight}
\pagestyle{empty}

\begin{document}
\noindent
Some auxiliary info regarding eqns. (4-1) -- (4-5). We note that the normal
derivative in (4-5) also is taken in direction of the outer normal
which is consistent with other use of the term ``normal'' in the text.
However, the apparent
lack of a ``minus''-sign in front of the $V\, \partial\,\Psi/\partial n$ term
when using $\Psi=\besY_0(kr)$ (because $\besY_0' = -\besY_1$) shall be explained below.

First we focus on (4-1) which connects the $\vec{H}$-field to
$\partial V/\partial n$:
\begin{displaymath}
    -j\omega\mu\vec{H} = \rot \vec{E}
  = \rot \left(-\,\ez\,\frac{V}{D}\right)
  = \frac{-1}{D} \nabt \times ( \ez\, V )
  = \frac{1}{D}\,\ez\, \times \nabt V
\end{displaymath}
and therefore (multiplying by $-\,\eg$ on both sides)
\begin{displaymath}
     j\omega\mu D\, (\eg\cdot\vec{H})
  = -\,\eg\cdot(\ez\,\times\nabt V)
  = -\nabt V \cdot (\eg\times\ez)
  = -\nabt V \cdot \en
  = -\frac{\partial V}{\partial n}
\end{displaymath}
hence
\begin{equation}
\label{eqn:eqn1}
    \frac{\partial V}{\partial n}
  = -j\omega\mu D\, (\eg\cdot\vec{H})
  = -j k \sqrt{\frac{\mu}{\varepsilon}} D\, (\eg\cdot\vec{H})
\end{equation}

Secondly, we calculate $\partial \besY_0(kr)/\partial n$ (the vector $\vec{r}$
pointing from field or ``test'' point $P$ towards the boundary or ``source''
point $Q$, hence 
$\partial/\partial r=\partial/\partial n$)
\begin{equation}
\label{eqn:eqn2}
	\frac{\partial \Psi}{\partial n}
  = \frac{\partial \besY_0(kr)}{\partial n}
  = \besY_0(kr)' k \frac{\partial r}{\partial n}
  = - \besY_1(kr) k \frac{\partial r}{\partial n}
\end{equation}

We now turn to Green's identity (4-4)
\begin{displaymath}
	\int\!\!\!\int_A\!\! \left(\Psi\Delta_t V - V \Delta_t \Psi\right)\, dA
	= \oint_\Gamma
		\left(
			\Psi \frac{\partial V}{\partial n}
	   -    V    \frac{\partial \Psi}{\partial n}
	    \right)
		\dd s
\end{displaymath}
and use $\Psi=\besY_0$. We work around the singularity at $kr=0$ by excluding
the field point $P$ from the surface area $A$ but add a little circle $\Gamma_P$
around $P$ to the contour on the right hand side. Since now both, $V$ and $\besY_0$
satisfy the wave equation (4-2) the left hand side vanishes everywhere:
\begin{displaymath}
	0 = \oint_{\Gamma + \Gamma_P}
		\left(
			\besY_0 \frac{\partial V}{\partial n}
	   -    V    \frac{\partial \besY_0}{\partial n}
	    \right)
		\dd s
\end{displaymath}
Substituting $\partial V/\partial n$  and $\partial \besY_0/\partial n$
using \eqnr{eqn1} and \eqnr{eqn2}, respectively we note that a minus sign
appears in both places so that
\begin{eqnarray*}
	0 & = & \oint_{\Gamma + \Gamma_P}
		\left(
		\besY_0(kr)
          \left(-j k \sqrt{\frac{\mu}{\varepsilon}} D\, (\eg\cdot\vec{H})\right)
	   -    V   
        \left( - \besY_1(kr) k \frac{\partial r}{\partial n} \right)
	    \right)
		\dd s
\\
	 & = & k \oint_{\Gamma + \Gamma_P}
		\left(
          j \sqrt{\frac{\mu}{\varepsilon}} D\, (\eg\cdot\vec{H})\, \besY_0(kr)
	   -    V\, 
         \besY_1(kr) \frac{\partial r}{\partial n}
	    \right)
		\dd s
\end{eqnarray*}
All that is left to do is calculating the contour integral along the small
circle $\Gamma_P$ of radius $r_\gamma$ for $\lim_{r_\gamma \rightarrow 0}$.
Because the inside area of $\Gamma_P$ is {\em outside} of $A$, the normal
derivative $\partial/\partial n = -\partial/\partial r_\gamma$.

Since $V$ is well-behaved around $P$, and
$\int_0^{2 \pi} \partial V /\partial r_\gamma \dd \varphi < C$ with
$C$ a suitable constant the first part vanishes:
\begin{eqnarray*}
\lim_{r_\gamma\rightarrow 0}
k \oint_{\Gamma_P}\!\!\! \frac{-\partial V}{\partial n} \besY_0(k r_\gamma) \dd s
 & = &
\lim_{r_\gamma\rightarrow 0}
\int_0^{2 \pi}\!\!\!
       \frac{\partial V}{\partial r_\gamma} \besY_0(k r_\gamma) k r_\gamma \dd \varphi
\\ & < &
\lim_{kr_\gamma\rightarrow 0}
C \besY_0(k r_\gamma) k r_\gamma
= 0
\end{eqnarray*}
For the second part we obtain
\begin{eqnarray*}
\lim_{r_\gamma\rightarrow 0}
k \oint_{\Gamma_P}\!\!\!
         V
         \besY_1(kr_\gamma) \frac{\partial r_\gamma}{\partial n}
		 \dd s
& = &
\lim_{r_\gamma\rightarrow 0}
- \int_0^{2 \pi}
	V
    \besY_1(kr_\gamma) \frac{\partial r_\gamma}{\partial r_\gamma}
	k r_\gamma
	\dd \varphi
\\ 
& = &
- 2 \pi 
\lim_{r_\gamma\rightarrow 0}
	V
    \besY_1(kr_\gamma)
	k r_\gamma
\\
& = & - 2 \pi \left(\frac{-2}{\pi}\right) V = 4 V
\end{eqnarray*}
yielding eq. (4-5) of the text.
\end{document}
